\documentclass[11pt]{article}
\usepackage[utf8]{inputenc}
\usepackage{ifpdf}
\ifpdf
    \usepackage[pdftex]{graphicx}   % to include graphics
    \pdfcompresslevel=9 
    \usepackage[pdftex,     % sets up hyperref to use pdftex driver
            plainpages=false,   % allows page i and 1 to exist in the same document
            breaklinks=true,    % link texts can be broken at the end of line
            colorlinks=true,
            pdftitle=My Document
            pdfauthor=My Good Self
           ]{hyperref} 
    \usepackage{thumbpdf}
\else
    \usepackage{graphicx}       % to include graphics
    \usepackage{hyperref}       % to simplify the use of \href
\fi

\title{Notas sobre TeXnicle}
\author{Armando Galve Carbó}
\date{}

\begin{document}
\maketitle
\section{Primeros pasos}
\subsection{Introducción} El editor parece más configurable que Latexian. 
Permite también la actualización de la vista mediante \textit{live update}

Una ventaja, que en determinados momentos puede ser interesante, es la de 
desactivar la vista en vivo. O que está se puede ir desarrollando en una ventana 
independiente. Dejamos una ventana pequeña para ver la estructura de la página y 
dejamos una ventana en segundo plano donde se puede ver con detalle el documento 
que resulta.

He probado el {\textit typeset} tanto con \LaTeX como con Xetex, con el que se 
usa {\textit fontspec}

Otro detalle muy bueno es el completado de código que hace que la escritura sea 
sorprendentemente rápida

\end{document}  
